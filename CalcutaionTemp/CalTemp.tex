% !TEX program = latexmk
\documentclass[b5paper,UTF8]{book}

% 参数
% #omega_0
% #mu_s
% #mu_z

\usepackage{ctex}
% plan
\usepackage{lscape}
\usepackage{geometry}
    \geometry{top = 0.8 in}
    \geometry{bottom = 1 in}
    \geometry{left = 0.5 in}
    \geometry{right = 0.5 in}
% headings and footings
\usepackage{fancyhdr}
    \pagestyle{fancy}
    \fancyhead{}
    % E even page O odd page 
    % L left C center R right 
    % H heading F footing
    \fancyhead[LO,RE]{{\arabic{chapter}}}
    \fancyhead[RO,LE]{\rightmark}
% Chinese title
\usepackage{titlesec}
% \titleformat{command}[shape]{format}{label}{sep}{before}[after]
\titleformat{\chapter}{\centering\Huge\bfseries}{第\,\thechapter\,章}{1em}{}
% formulas
\usepackage{amsmath}
% table
\usepackage{booktabs}
% pictures
% \usepackage{graphicx}
% \usepackage{subfigure}
\usepackage{tkz-graph}
% flow-chart
\usepackage{tikz,mathpazo}
\usetikzlibrary{shapes.geometric, arrows}



% document

\begin{document}

% cover

\thispagestyle{empty}

\begin{center}
\vspace*{3 cm}

\includegraphics[width = 3 in]{cover.png}
\vspace*{1 cm}

\heiti{\Huge{《偶学费是个牛逼》学习笔记}}
\vspace*{1 cm}

\Large {1218年3月21日}
\vspace*{0.5cm}

梅超风
\end{center}
\vspace*{4cm}

\clearpage

% 封二
\newpage
\thispagestyle{empty}

\noindent
书名:《九阴真经》\\
作者:黄裳\\
出版:大宋皇家出版社\\
版次:1164年8月第1版\\
书号:ISBN 7-181-9527

\clearpage

% 目录
\setcounter{page}{1}
\pagenumbering{Roman}
% arabic - 阿拉伯数字
% roman - 小写的罗马数字
% Roman - 大写的罗马数字
% alph - 小写的字符形式
% Alph -大写的字符形式
\tableofcontents
% \thispagestyle{empty}

\clearpage

% 目录之后开始编页码
\setcounter{page}{1}
\pagenumbering{arabic}
% arabic - 阿拉伯数字
% roman - 小写的罗马数字
% Roman - 大写的罗马数字
% alph - 小写的字符形式
% Alph -大写的字符形式


\chapter{数学公式}

\section{行内和行间数学式子}

在此之前说明以下符号的应用,
字体大小的双斜杠:$ \verb|\\|$,
比较大一号的双斜杠:$\backslash\backslash$。

下面来看一下简单的行内式子和行间式子。

基本风压:$\omega_0 = 0.4$,
% 基本风压:$\omega_0 = #omega_0$,
高度变化修正系数:$\mu_z = 1$,
% 高度变化修正系数:$\mu_z = #mu_z$,
体型系数:$\mu_s = 1$,
% 体型系数:$\mu_s = #mu_s$,
以下是不带编号的公式:

\[ \omega_k = \beta_z \mu_z \mu_s \omega_0 \]

\hangafter = 1
\hangindent = 2 em
式中:\\
$\omega_0$---基本风压\\
$\mu_z$---高度变化修正系数\\
$\mu_s$---体型系数\\
$\omega_k$---风压标准值\\



这个时候我要写很多废话来把这里凑满一行内容。
我现在写的字还不够一行,所以我还要多写点。
同样不带编号的式子,
但是有多行的情况:

\begin{align*}
    f(x) &= (x+a)(x+b) \\
    &= x^2 + (a+b)x + ab
\end{align*}


\section{单个公式}



\subsection{标准单个公式环境}

\begin{verbatim}
\begin{equation}
...
\end{equation} 
\end{verbatim}

它是最一般的公式环境,表示一个公式,
默认情况下之表示一个单行的公式,
但是它的功能可以通过内嵌各种其他环境进行扩展。
它可以内嵌的一些关于对齐的环境将在后面介绍。
然后来一个代编号的公式。
\begin{equation}\label{eq:wind}
    \omega_0 = \beta_z \mu_z \mu_s \omega_k \\
            = 0.4 \times 1 \times 1 \times 0.4
\end{equation}
这个时候我们要引用公式\ref{eq:wind},
来说明引用公式的一些问题。


\section{多个公式}

\subsection{公式跨页}

默认一个公式环境里面的多行公式是不会跨页显示的,
可以使用下面命令开启全局的自动跨页显示:
\begin{verbatim}
\allowdisplaybreaks
或
\allowdisplaybreaks[n]
\end{verbatim}
n的值为0到4,表示分页的坚决程度,
例如0表示能不分页就不分页,4表示强制分页。

也可以在公式环境中使用
\begin{verbatim}
\displaybreak 
\end{verbatim}

手工指定分页,它同样可以带有这个参数。

\subsection{多个公式基本的对齐环境}

align(多个公式)
这是最基本的对齐环境,
其他多公式环境都不同程度地依赖它。
与表格环境一样,
它采用“\&”分割各个对齐单元,
使用“$ \verb|\\|$”换行。
它的每行是一个公式,都会独立编号。
在排版过程中,它将\&分出来的列又分成组,
组间特定方式排版,具体方式在flalign环境中讨论。

\begin{align}
 f(x) &= (x+a)(x+b) \\
 &= x^2 + (a+b)x + ab
\end{align}

gather
它是最简单的多行公式环境,
自己不提供任何对齐。
其中的各行公式按照全局方式分别对齐。
在设置了全局左对齐之后,
因为不存在内部各个公式之间对排版的干扰,
这种环境非常适合写数学推导或者证明。

\begin{gather}
E(X)=\lambda	\qquad	D(X)=\lambda	\\
E(\bar{X})=\lambda	\\
D(\bar{X})=\frac{\lambda}{n}	\\
E(S^2)=\frac{n-1}{n}\lambda
\end{gather}



flalign
虽然可以使用多个\&,
但是比较一般的用法是在只在等号前面使用一个\&,
它使所有列表现地像是根据等号对齐了,
因此这个环境很适合用来编写多行的公式推导和数值计算过程。
例如:


\begin{align*}
E(S^2)	&=E\left(\frac{1}{n} \Sigma (X_i-\bar{X})^2\right)	\\
&	=E\left(\frac{1}{n}\Sigma X_i^2\right)
 - E\left(\frac{1}{n}\Sigma 2\bar{X}X_i\right)
 + E\left(\frac{1}{n}\Sigma \bar{X}^2\right)	\\
&	=EX^2 -E(\bar{X}^2)	\\
&	=DX + (EX)^2 - D\bar{X} - (E\bar{X})^2	\\
&	=\frac{n-1}{n}DX
\end{align*}


面完整地描述一下flalign和align环境是如何处理多\&情况下的对齐的。

根据\& (假设n个)将一行分为n+1列。
从左向右将列两个分为一组,第一组紧靠页左侧,
最后一组紧靠页左侧,其余均匀散布在整个行中。
当公式比较短时,中间可能会有几段空白。

需要注意的是:
\begin{enumerate}
\item 每一组内部也是有对齐结构的!
    它们在所在位置上向中间对齐的,
    即第一列向右对齐,第二列向左对齐。
\item 所谓紧靠页左/右是在进行了组内对齐调整之后,
    最长的一块紧靠上去。
    也就是说对于长度不一两行,
    较短的那一行是靠不上去的。
\item 如果总共有奇数个列,
    及最后一组只有一个列,
    则它右对齐到页右侧,
    即所有行的最后一列的右侧都靠在页右侧。
\end{enumerate}



下面用一个极端的例子来说明这个环境的特点:
\begin{flalign}
xyz &= b+c &=123 &=123 &=ssss \\
x  &= 1+12 &=432523452345 &=2 &=a\\
  &= 2  &=982739 &=p &=ttt
\end{flalign}

multline
注意是multline 不是multiline,虽然它就是那个意思。
它不支持“\&”分列。首行左对齐,末行右对齐,其余各行分别按照全局方式对齐。


\begin{multline}
E(X)=\lambda	\qquad	D(X)=\lambda	\\
E(\bar{X})=\lambda	\\
D(\bar{X})=\frac{\lambda}{n}	\\
E(S^2)=\frac{n-1}{n}\lambda
\end{multline}


没有常见的应用模式。

alignat
它接收一个参数用来指定根据哪一列对齐。

\begin{alignat}{2}
 \sigma_1 &= x + y  &\quad \sigma_2 &= \frac{x}{y} \\  
 \sigma_1' &= \frac{\partial x + y}{\partial x} & \sigma_2'
    &= \frac{\partial \frac{x}{y}}{\partial x}
\end{alignat}

\subsection{用于内嵌的对齐环境}

这些环境无法独立构成一个数学环境,必须要嵌入在其他环境内部。

例如:aligned
\begin{equation}
 \left.\begin{aligned}
        B'&=-\partial \times E,\\
        E'&=\partial \times B - 4\pi j,
       \end{aligned}
 \right\}
 \qquad \text{Maxwell's equations}
\end{equation}

$\backslash$left
和
$\backslash$right
后加一个括号的
表示用于自动调整各种括号的大小,
必须配对使用。
公式中的
$\backslash$left. 
是一个虚的
$\backslash$left,
目的是为了和
$\backslash$right
$\backslash$\}配对。

split
它用于将一个公式拆分成多行,
但是它整体还只是一个公式。

\begin{equation}
\begin{split}
(a + b)^4
&= (a + b)^2 (a + b)^2      \\
&= (a^2 + 2ab + b^2)(a^2 + 2ab + b^2)  \\
&= a^4 + 4a^3b + 6a^2b^2 + 4ab^3 + b^4
\end{split}
\end{equation}



\section{计算过程}

有时候我们并不需要公式,
而是要列出我们的计算过程,
这个时候除了写出公式和取值以外,
还需要列出大量的中间过程。
这个时候最好能用行间公式,
并且书写的时候左对齐。



\begin{align*}
    A &= b + c(d + e) \\
    &= 25 + 95 \times (65 + 56)  \\
\end{align*}













\chapter{作图技术}

\section{插入图片}

\begin{figure}[htbp]%[htbp] 
    \begin{center} 
        \includegraphics[width = 0.5 \textwidth]{cover.png}
        % \includegraphics{bibliography.jpg} 
        \caption{文献引用} 
        \label{fig:bibliography} 
    \end{center} 
\end{figure}




\section{流程图}

% 流程图定义基本形状
\tikzstyle{startstop} = [rectangle, rounded corners, minimum width=3cm, minimum height=1cm,text centered, draw=black, fill=red!30]
\tikzstyle{io} = [trapezium, trapezium left angle=70, trapezium right angle=110, minimum width=3cm, minimum height=1cm, text centered, draw=black, fill=blue!30]
\tikzstyle{process} = [rectangle, minimum width=3cm, minimum height=1cm, text centered, draw=black, fill=orange!30]
\tikzstyle{decision} = [diamond, minimum width=3cm, minimum height=1cm, text centered, draw=black, fill=green!30]
\tikzstyle{arrow} = [thick,->,>=stealth]

\begin{tikzpicture}[node distance=2cm]
%定义流程图具体形状
\node (start) [startstop] {Start};
\node (in1) [io, below of=start] {Input};
\node (pro1) [process, below of=in1] {Process 1};
\node (dec1) [decision, below of=pro1, yshift=-0.5cm] {Decision 1};
\node (pro2a) [process, below of=dec1, yshift=-0.5cm] {Process 2a};
\node (pro2b) [process, right of=dec1, xshift=2cm] {Process 2b};
\node (out1) [io, below of=pro2a] {Output};
\node (stop) [startstop, below of=out1] {Stop};
%连接具体形状
\draw [arrow](start) -- (in1);
\draw [arrow](in1) -- (pro1);
\draw [arrow](pro1) -- (dec1);
\draw [arrow](dec1) -- (pro2a);
\draw [arrow](dec1) -- (pro2b);
\draw [arrow](dec1) -- node[anchor=east] {yes} (pro2a);
\draw [arrow](dec1) -- node[anchor=south] {no} (pro2b);
\draw [arrow](pro2b) |- (pro1);
\draw [arrow](pro2a) -- (out1);
\draw [arrow](out1) -- (stop);
\end{tikzpicture}

\section{简图}

\begin{tikzpicture}[scale = 1]
    \draw (0, 0) rectangle (2.5, 6.0);
    \draw (0.405, 0.405) circle (0.125);
    \draw (0.968333333333, 0.405) circle (0.125);
    \draw (1.53166666667, 0.405) circle (0.125);
    \draw (2.095, 0.405) circle (0.125);
\end{tikzpicture}









\chapter{表格}


湛江网架参数:高度,140.000m;半径,20m。    
汉中(陶修)网架参数:高度,43777;半径,62900。
岩井网架参数:高度,140.000m;半径,20m。


\begin{table}[htbp]
\centering
\caption{网壳划分形式}
\label{tab:TrussDivided}
\begin{tabular}{cc}
\toprule
    百分比            & 轴线含钢量     \\
\midrule
双层凯威特A-6  8.54   &      20\%      \\
双层凯威特A-6  8.54   &      20\%      \\
双层凯威特A-6  6.40   &      20\%      \\
双层凯威特A-6  6.27   &      20\%      \\
\hline
湛江                  &      20\%      \\
汉中                  &      20\%      \\
汉中                  &      20\%      \\
\bottomrule                             
\end{tabular}
\end{table}


\begin{landscape}
% \usepackage{lscape} % 单页的页面布局
\begin{table}
\centering
\caption{网壳位移}
\label{tab:TrussDisplacement}
\begin{tabular}{ccc}
\toprule
                & 支座刚度(m) & \\
\midrule 
网架     & 3            &       \\
网架     & 0            &       \\
网架     & 2.8 2.8   &          \\
修哥网架 & 2.8 2.8   &          \\
湛江网架 &               &      \\
\bottomrule                             
\end{tabular}
\end{table}
\end{landscape}     



% Courier New
% monospace
% Source Code Pro
% Menlo
% Consolas
% Monaco
% Dejavu Sans Mono






\chapter{文献引用}

引用一篇文章\cite{article1},引用一本书\cite{book1},等等。

% LaTeX 标准选项及其样式共有以下8种:

% plain,按字母的顺序排列,比较次序为作者、年度和标题.
% unsrt,样式同plain,只是按照引用的先后排序.
% alpha,用作者名首字母+年份后两位作标号,以字母顺序排序.
% abbrv,类似plain,将月份全拼改为缩写,更显紧凑.
% ieeetr,国际电气电子工程师协会期刊样式.
% acm,美国计算机学会期刊样式.
% siam,美国工业和应用数学学会期刊样式.
% apalike,美国心理学学会期刊样式.

\begin{thebibliography}{99} 

\bibitem{article1} 文章标题 作者 期刊 年代 页码 
% \bibitem{book1} 书标题 作者 出版社 年代 
\bibitem{book1} 结构力学I(第二版),萧允徽,张来仪,机械工业出版社,2015.8 
    
\end{thebibliography} 



\end{document}
